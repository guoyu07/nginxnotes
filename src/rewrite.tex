\part{URL Rewrite}



\chapter{Overview}


URL重写 (URL Rewriting) 是一种REST的相关技术。

在 Web服务器中,URL重写可以针对用户所提供的 URL 进行转换后,再传入 Web服务器中的程序处理器。

\begin{compactitem}
\item mod\_rewrite、mod\_alias(Apache)
\item HttpRedirectFilter、UrlRewriteFilter(J2EE)
\item mod\_rewrite(Lighttpd)
\end{compactitem}

URL重写最常见的用法就是将一组 URL 层次结构字符串,转换成带有 query string 的 URL,或是反向转换。

\begin{compactitem}
\item \url{http://www.somebloghost.com/Blogs/Posts.php?Year=2006\&Month=12\&Day=10}经过 URL 重写可以变成\url{http://www.somebloghost.com/Blogs/2006/12/10/}

\item \url{http://www.somehost.com/Blogs/2006/12/}经过URL重写可以变成\url{http://www.somehost.com/Blogs.aspx?year=2006\&month=12}
\end{compactitem}

URL重写可以让用户使用较直觉的方式来输入 URL(这也是 REST 的主要目的),也是搜索引擎最优化(SEO)的作法之一,同时应用程序开发者可以利用URL重写机制来将参数隐藏起来,尽量避免让网络上的恶意用户收集到有利于发动攻击的信息。





