\part{Core}


\chapter{Overview}


\begin{lstlisting}
user www www;
error_log /var/log/nginx/error.log info;
pid /var/run/nginx.pid;

worker_processes 4;
worker_priority 0;
worker_cpu_affinity 0001 0010 0100 1000;
#worker_rlimit_core 
#worker_rlimit_nofile
#working_directory

daemon on;
master_process on;
debug_points stop;

env MALLOC_OPTIONS;
env PERL5LIB=/data/site/modules;
env OPENSSL_ALLOW_PROXY_CERTS=1;

#ssl_engine 

#thread_pool default threads=32 max_queue=65536;
timer_resolution 100ms;

load_module modules/ngx_mail_module.so;

lock_file logs/nginx.lock;

pcre_jit on;                                                                                                                                                                                                                                                                                                                                                                                                                                                                                                                                                                                                                                                                                                                                                                                                                                                                                                                                                                                                                                                                                                                                                                                                                                                                                                                                                                                                                                                                                                                                                                                                                                                                                                                                                                                                                                                                                                                                                                                                                                                                                                                                                                                                                                                                                                                                                                                                                                                                                                                                                                                                                                                                                                                                                                                                                                                                                                                                                                                                                                                                                                                                                                                                                                                                                                                                                                                                                                                                                                                                                                                                                                                                                                                                                                                                                                                                                                                                          

include /etc/nginx/modules-enable/*.conf;

events {
  use epoll;
  worker_connections 1024;
  worker_aio_requests 32;
  
  
  # multi_accept on;
  accept_mutex off;
  accept_mutex_delay 500ms;
  debug_connection 127.0.0.1;
}

http {
  # ...
}

mail {
  # ...
}

stream {
  # ...
}

google_perftools_profiles /path/to/profile;
\end{lstlisting}

Nginx的核心指令(core directive)如下:

\begin{compactitem}
\item accept\_mutex(串行执行worker来接收请求)
\item accept\_mutex\_delay(串行执行worker延迟)
\item daemon(持久执行)
\item debug\_connection
\item debug\_points
\item error\_log
\item env
\item events
\item include
\item load\_module
\item lock\_file
\item master\_process
\item multi\_accept
\item pcre\_jit
\item pid
\item ssl\_engine(SSL硬件加速器)
\item thread\_pool
\item timer\_resolution
\item use
\item user
\item worker\_aio\_requests
\item worker\_connections
\item worker\_cpu\_affinity
\item worker\_priority
\item worker\_processes
\item worker\_rlimit\_core
\item worker\_rlimit\_nofile
\item working\_directory
\end{compactitem}



\chapter{accept\_mutex}

当一个新连接到达时,如果激活了accept\_mutex,那么多个Worker将以串行方式来处理,其中有一个Worker会被唤醒,其他的Worker继续保持休眠状态;如果没有激活accept\_mutex,那么所有的Worker都会被唤醒,不过只有一个Worker能获取新连接,其它的Worker会重新进入休眠状态,这就是「惊群问题」。

假设你养了一百只小鸡,现在你有一粒粮食,那么有两种喂食方法:

\begin{compactitem}
\item 你把这粒粮食直接扔到小鸡中间,一百只小鸡一起上来抢,最终只有一只小鸡能得手,其它九十九只小鸡只能铩羽而归。这就相当于关闭了accept\_mutex。
\item 你主动抓一只小鸡过来,把这粒粮食塞到它嘴里,其它九十九只小鸡对此浑然不知,该睡觉睡觉。这就相当于激活了accept\_mutex。
\end{compactitem}

Nginx缺省激活了accept\_mutex,虽然说不会有惊群问题,但是这是一种保守的选择。

Apache动辄就会启动成百上千的进程,如果发生惊群问题的话,影响相对较大;但是对Nginx而言,一般来说,worker\_processes会设置成CPU个数,所以最多也就几十个,即便发生惊群问题的话,影响相对也较小。

即使关闭了accept\_mutex,仍然可能会引起一定程度的惊群问题,表现为上下文切换增多(sar -w)或者负载上升,如果网站访问量比较大,为了系统的吞吐量,还是建议大家关闭accept\_mutex。

修改一下问题的场景,不再只有一粒粮食,而是一盆粮食时,如果仍然采用主动抓小鸡过来塞粮食的做法就太低效了,一盆粮食不知何年何月才能喂完,可以设想一下几十只小鸡排队等着喂食时那种翘首以盼的情景,此时更好的方法是把这盆粮食直接撒到小鸡中间,让它们自己去抢,虽然这可能会造成一定程度的混乱,但是整体的效率无疑大大增强了。



和激活accept\_mutex 相比,关闭 accept\_mutex 的时候,请求在多个 worker 间的分配更均衡。

\begin{compactitem}
\item 开启 accept\_mutex 时:

\begin{lstlisting}[language=bash]
shell> ./ngx-req-distr -m `cat /path/to/nginx.pid`
Tracing 12970 12971 12972 12974 (/path/to/nginx)...
Hit Ctrl-C to end.
^C
worker 12970:	0 reqs
worker 12971:	37 reqs
worker 12972:	127 reqs
worker 12974:	3 reqs
\end{lstlisting}

\item 关闭 accept\_mutex 时:

\begin{lstlisting}[language=bash]
shell> ./ngx-req-distr -m `cat /path/to/nginx.pid`
Tracing 20433 20434 20435 20436 (/path/to/nginx)...
Hit Ctrl-C to end.
^C
worker 20433:	75 reqs
worker 20434:	32 reqs
worker 20435:	29 reqs
worker 20436:	44 reqs
\end{lstlisting}

\end{compactitem}

\chapter{accept\_mutex\_delay}

在设置了 \texttt{accept\_mutex on;}后,最好设置下\texttt{accept\_mutex\_delay 50ms;}。



\chapter{daemon}


\chapter{debug\_connection}



\chapter{debug\_points}


\chapter{error\_log}


\chapter{env}


\chapter{events}


\chapter{include}



\chapter{load\_module}


\chapter{lock\_file}



\chapter{master\_process}


\chapter{multi\_accept}


multi\_accept 告诉 Nginx 收到一个新连接通知后接受尽可能多的连接。


\chapter{pcre\_jit}


\chapter{pid}


\chapter{ssl\_engine}


\chapter{thread\_pool}


\chapter{timer\_resolution}


\chapter{use}


\chapter{user}


\chapter{worker\_aio\_requests}


\chapter{worker\_connections}


\chapter{worker\_cpu\_affinity}


\chapter{worker\_priority}


\chapter{worker\_processes}


\chapter{worker\_rlimit\_core}


\chapter{worker\_rlimit\_nofile}


\chapter{working\_directory}


\chapter{http}


\chapter{mail}


\chapter{stream}

