\part{Intro}


\chapter{Overview}

\section{Nginx}


Nginx是一个用于反向代理HTTP, HTTPS, SMTP, POP3, IMAP的协议链接的Web服务器软件,同时还是一个负载均衡器和一个HTTP缓存服务器。


作为一个面向性能设计的HTTP服务器,Nginx相比Apache、lighttpd具有占有内存少,稳定性高等优势。例如,与旧版本(<=2.2)的Apache不同,nginx不采用每客户机一线程的设计模型,而是充分使用异步逻辑,削减了上下文调度开销,所以并发服务能力更强。

Nginx整体采用模块化设计,其http服务器核心功能也是一个模块。


Nginx有丰富的模块库和第三方模块库,配置灵活。 以前,旧版本的nginx的模块是静态的,添加和删除模块都要对nginx进行重新编译,1.9.11以及更新的版本已经支持动态模块加载。


在Linux操作系统下,nginx使用epoll事件模型,得益于此,nginx在Linux操作系统下效率相当高。例如,Nginx在官方测试的结果中能够支持五万个并行连接,而在实际的运作中,可以支持二万至四万个并行链接。


Nginx在OpenBSD或FreeBSD操作系统上采用类似于epoll的高效事件模型kqueue。

Nginx和PHP-FPM的组合是一种稳定、高效的PHP运行方式,效率要比传统的Apache和mod\_php高出不少。其中,PHP-FPM从PHP5.3.3开始合并到PHP核心,编译时加上\texttt{-\/-enable-fpm}即可提供支持。

在PHP-FPM和Nginx集成使用时,PHP-FPM以守护进程在后台运行,Nginx响应请求后,自行处理静态请求,PHP请求则经过fastcgi\_pass交由PHP-FPM处理,处理完毕后返回。

如果PHP-FPM崩溃退出,那么前端将失去响应,这时Nginx会返回“The page you are looking for is temporarily unavailable. Please try again later.”的错误信息。


\section{Tengine}

Tengine是一个由淘宝从nginx fork出来的HTTP服务器,同时Tengine也不断从Nginx继承其更新,所以Tengine兼容Nginx最新版1.6.2的所有特性,与Nginx的配置兼容。

Tengine从原来的nginx添加了下列各项内容:

\begin{compactitem}
\item 通过对上传到HTTP后端服务器或FastCGI服务器的请求整流,以及通过增加一致性hash模块、会话保持模块,加上对服务器的主动健康检查,根据服务器状态而自动加添或减少服务器的实例,大量减少对服务器机器的I/O压力,大大增强其负载均衡能力;

\item 支持动态模块加载(DSO)支持,通过把模块编译成为可共享程序库,令服务器增添模块后无需再把整个服务器程序重新编译;
\item 受到Apache HTTP Server的modconcat功能启导的CONCAT模块,可组合多个CSS、JavaScript文件的访问请求变成一个请求,以减少数据流量及提高压缩比;
\item 输入过滤器主体,以更方便地管理在防火墙和事件到HTTP级别之间的连接;
\item 模块Sysguard,限制使用的存储器或CPU资源时使用率超过某个阈值。

\end{compactitem}

上列内容主要是从处理请求的效率及扩展性的增润,而且这些修正部分已为nginx主分支接纳。

\begin{lstlisting}[language=bash]
$ sudo add-apt-repository ppa:brightbox/tengine
$ sudo apt-get update
$ sudo apt-get install tengine
\end{lstlisting}





